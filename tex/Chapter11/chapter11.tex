\chapter{sheet-11 : Quantum Dynamics}

\section{Quantum Dynamics}

\subsection{Schrodinger picture}

The basic question of relativistic quantum dynamics is given an initial state $\ket{\psi(t_0)} $ of the system, how the state at time t, $\ket{t} $, is determined. The assertion that $ \ket{\psi(t_0)}$ determines $\ket{\psi(t)}$ is the quantum mechanical form of the principle of causality, and we shall assume it.\\
In addition, we postulate an extension of the principle of superposition to include the temporal development of states. This states that if $\ket{\psi_1(t_0)}$  and $\ket{\psi_2(t_0)}$ seperately evolve into $\ket{\psi_1(t}$ and $\ket{\psi_1(t)}$, Then a superposition
\[ \ket{\psi(t_0)} = \lambda_1 \ket{\psi_1(t_0)} + \lambda_2 \ket{\psi_2(t_0)}\] \\
develops into \\
\[ \ket{\psi(t)} = \lambda_1 \ket{\psi_1(t)} + \lambda_2 \ket{\psi_2(t)}\]\\
i.e each component of the state moves independently of each other. This means that $\ket{t}$ can be obtained from an arbitrary initial state by the application of linear operator:\\
\[\ket{\psi (t)}= T(t, t_0) \ket{\psi (t_0)}\] \\
The operator T is called the time evolution operator for quantum mechanical state vectors

\subsection{Schrodinger equation}

The exact form of the time evolution operator can be found from the schrodinger equation, which is a postulate of quantum mechanics describing how the state vector changes with time. The schrodinger equation is \\
\[i \hbar \frac{\partial}{\partial t} \ket{\psi t} = H \ket{\psi t} \]\\

where H is a linear operator, called the Hamiltonian of the system.
