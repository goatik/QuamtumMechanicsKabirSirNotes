\chapter{sheet-12 : The Path Integral Formulation of Quantum Theory}
\label{chapter1.path-intigral}

The path integral formultion of quantum theory

\subsection{Background materials:}

Basis states
		\[\hat{O_s} \ket{q} = q \ket{q} \]
		\[\hat{P_s} \ket{q} = p \ket{p} \]
		
	The states {$\ket{q}$} and {$\ket{p}$} qre basis states i.e. \\
	\[\int dq \ket{q}\bra{q} = \mathbb{1} \]
	\[\int \ket{p} \bra{p}  = \mathbb{1} \]
	where the normalization is chosen as 
	\[\bra{q}\ket{q'} = \delta(q-q') \]
	\[\bra{p}\ket{p'} = \delta(p-p') \]
	
	The operators $\hat{Q_s}$ and $\hat{Q_s}$ can be expressed in coordinate representation as follows:
	
	\[\bra{q}\hat{Q_s} = q\bra{q} \]
	\[\bra{q}\hat{P_s} = -i\hbar \pdv{q}  \bra{q} \]
	
	In momentum representation we have 
	\[\bra{p}\hat{Q_s}= i \hbar \pdv{p}\bra{p}   \]
	\[\bra{p}\hat{P_s}= p\bra{p} \]
	
	The fundamental commutation relation between $\hat{Q_s}$ and $\hat{P_s}$ is 
	\[ \comm{\hat{Q_s}}{\hat{P_s}} = i\hbar \mathbb{1} \]
	
	For later purpose we will need the momentum eigenstates in coordinate representation, i.e, $\bra{q}\ket{p}$, to find $\bra{q}\ket{p}$, we proceed as follows:
	\[\hat{P_s}\ket{p} = p\ket{p} \]
	or, \[ \matrixel{q}{\hat{P_s}}{p} = p \ip{q}{p}   \]
or \[ -i \hbar \pdv \ip{q}{p} \]
